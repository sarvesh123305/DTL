\documentclass[11pt,a4paper]{report}
\usepackage[utf8]{inputenc}
\usepackage{amsmath}
\usepackage{amsfonts}
\usepackage{amssymb}
\usepackage{graphicx}
%\usepackage{hyperref}
\usepackage[colorlinks=true]{hyperref}
\usepackage[left=2cm,right=2cm,top=2cm,bottom=2cm]{geometry}
\author{Sarvesh Anant Kulkarni}
\hypersetup{
    colorlinks = true,
    urlcolor = {blue},
}

\title{Biblography}
\begin{document}
\maketitle

1. Simple Method\\
~\cite{sar}
2. BiBtex Method\\

Moving Average Crossover: After graphing, two 
moving averages based on separate time periods tend to cross, 
which is known as a moving-average crossover ~\cite{abc}. A quicker 
moving average and a slower moving average are used in this 
indication (or more). The shorter moving average (short-term) ~\cite{pqr}
can be 5, 10, or 15 days, while the longer-term moving ~\cite{aa,sarvesh}
average might be 100, 200, or 250 days. Since it only 
evaluates prices over a short period of time, a short-term 
moving average is speedier and more responsive to daily 
price changes ~\cite{sarvesh,soniya}


\begin{thebibliography} {}

\bibitem{aa} Leslie Lamport, “LaTeX: A document preparation system”, Users guide and reference manual, 2nd Edition, 1994, by Addison-Wesley Professional. ISBN 0201529831

\bibitem{sarvesh}Stefan Kottwitz, “LaTeX Beginner's Guide: Create High-quality and Professional-looking Texts, Articles, and Books for Business and Science Using LaTeX, Packt Publishing, 2011. ISBN: 1847199860, 9781847199867” 
\href{https://www.latex-project.org/}{https://www.latex-project.org/}

\bibitem{pqr} Introduction to LaTEX, MIT
\href{http://web.mit.edu/rsi/www/pdfs/new-latex.pdf}{http://web.mit.edu/rsi/www/pdfs/new-latex.pdf}

\bibitem{soniya}A simple guide to LaTeX - Step by Step
\href{https://www.latex-tutorial.com/tutorials}{https://www.latex-tutorial.com/tutorials}

\bibitem{abc}[7] M. Young, The Technical Writer’s Handbook. Mill Valley, CA: Univer-
sity Science, 1989

\bibitem{sar} Wiki

\end{thebibliography} 

\end{document}